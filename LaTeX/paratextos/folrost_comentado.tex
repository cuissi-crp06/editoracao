% ESPAÇAMENTO VERTICAL ELÁSTICO NO TOPO DA PÁGINA
% \vspace*{\fill} cria um espaço que se expande para empurrar o conteúdo para baixo
% O asterisco (*) garante que o espaço seja mantido mesmo no topo de uma página
\vspace*{\fill}

% INÍCIO DO LAYOUT EM DUAS COLUNAS PARA O TÍTULO
% Primeira coluna: 30% da largura da linha de texto
\begin{minipage}{0.3\textwidth}
    % Comando que empurra todo o conteúdo desta coluna para a direita
    % Como a coluna está vazia, serve apenas para criar espaço em branco à esquerda
    \hfill
\end{minipage}%
% Segunda coluna: 70% da largura da linha de texto
% O % no final evita espaço indesejado entre as colunas
\begin{minipage}{0.7\textwidth}
% AMBIENTE DE ALINHAMENTO À ESQUERDA
\begin{flushleft}
% CONFIGURAÇÃO DE FONTE E COR PARA O TÍTULO PRINCIPAL
\fininha           % Comando personalizado que define peso da fonte (fino)
\LARGE             % Tamanho de fonte grande
\color{crp2}       % Cor personalizada "crp2" (provavelmente azul ou similar)
CARTILHA PARA\\    % Primeira linha do título com quebra manual (\\)
PRODUÇÃO DE\\      % Segunda linha do título

% MUDANÇA DE PESO E COR PARA A PARTE PRINCIPAL DO TÍTULO
\media             % Comando personalizado que define peso médio da fonte
\color{crp1}       % Muda para cor "crp1" (provavelmente a cor principal do CRP)
\textbf{DOCUMENTOS ESCRITOS\\    % \textbf{} deixa em negrito + quebra de linha
ANTICAPACITISTAS}  % Continua em negrito

% ESPAÇAMENTO VERTICAL ENTRE TÍTULO E INFORMAÇÕES DE PUBLICAÇÃO
\vspace*{2\baselineskip}  % Adiciona espaço equivalente a 2 linhas de texto
                          % \baselineskip = altura padrão de uma linha

% CONFIGURAÇÃO PARA AS INFORMAÇÕES DE PUBLICAÇÃO
\color{crp3}       % Muda para cor "crp3" (provavelmente uma cor mais sutil)
\Light             % Comando personalizado para fonte light/leve
\normalsize        % Retorna ao tamanho normal de fonte

% INFORMAÇÕES DE PUBLICAÇÃO (cada linha quebrada manualmente)
1º EDIÇÃO\\        % Número da edição
SÃO PAULO\\        % Cidade de publicação
2025               % Ano de publicação (sem \\ pois é a última linha)

\end{flushleft}    % Fecha o ambiente de alinhamento à esquerda
\end{minipage}     % Fecha a segunda coluna

% ESPAÇAMENTO ENTRE O TÍTULO E O LOGO
\vspace*{\baselineskip}  % Adiciona espaço de uma linha

% SEÇÃO CENTRALIZADA PARA O LOGO
\begin{center}
    % LOGO/ASSINATURA COMENTADO - VERSÃO EM TEXTO
    % Este bloco está todo comentado (%) - seria uma versão em texto do nome da instituição
    % Provavelmente foi substituído pelo logo em imagem abaixo
    %\color{crp1}\Light\textit{C\small%    % "C" em itálico, fonte light, cor crp1, tamanho pequeno
    %onselho }%                             % Resto da palavra "Conselho" em fonte normal
    %\color{crp1}\normalsize\media\textit{Regional de }%  % "Regional de" em itálico, peso médio
    %\color{crp1}\normalsize\pesada\textit{PSICOLOGIA }%  % "PSICOLOGIA" em itálico, peso pesado
    %\color{crp1}\normalsize\media\textit{SP}%            % "SP" em itálico, peso médio
    
    % LOGO EM IMAGEM (VERSÃO ATUAL SENDO USADA)
    \includegraphics[width = 0.75\linewidth]{assinatura_crp.png}
    % \includegraphics = comando para inserir imagens
    % [width = 0.75\linewidth] = define largura como 75% da largura da linha
    % {assinatura_crp.png} = nome do arquivo de imagem
\end{center}       % Fecha o ambiente centralizado
