% SEÇÃO: EQUIPE DE ORIENTAÇÃO E FISCALIZAÇÃO
% Título da primeira equipe técnica
\CreditoInstitucionalDois{Equipe de Orientação e Fiscalização}

% LAYOUT EM DUAS COLUNAS (PADRÃO USADO EM TODO O DOCUMENTO)
% Primeira coluna: 3cm de espaço vazio (margem/alinhamento)
\begin{minipage}{3cm}
{\null}  % Espaço completamente vazio
\end{minipage}%
% Segunda coluna: restante da largura da linha
\begin{minipage}{\linewidth-3cm}
\begin{CreditoInstitucional}
% CARGOS DE LIDERANÇA DA EQUIPE
\CargoNosCreditos{%
    Lauren Mariana Mennocchi & \textbf{gerente\newline Técnico-Política}\\
    % \newline força quebra de linha dentro da célula da tabela
    % Diferente de \\, o \newline quebra dentro do próprio campo
    Andrea Gobato Quintavalle & \textbf{coordenadora}\\}
    
% DEMAIS MEMBROS DA EQUIPE (segundo bloco)
\CargoNosCreditos{%
    Gabriele da Silva Freire e\newline           % Nomes múltiplos na mesma função
    Rafael Santos Barboza	& especialista técnica/o — psicóloga/o\\
    % Note o uso de "e\newline" para listar duas pessoas na mesma função
    
    Edileine Gomes Marchewsky e\newline          % Novamente dois nomes
    Regina Sampaio Lott	& profissionais de suporte\newline administrativo\\
    % O cargo também usa \newline para quebrar "suporte" e "administrativo"
    }
\end{CreditoInstitucional}
\end{minipage}

% SEÇÃO: EQUIPE DE COMUNICAÇÃO
\CreditoInstitucionalDois{Equipe de Comunicação}

% MESMO PADRÃO DE LAYOUT EM DUAS COLUNAS
\begin{minipage}{3cm}
{\null}
\end{minipage}%
\begin{minipage}{\linewidth-3cm}
\begin{CreditoInstitucional}
% LIDERANÇA DA EQUIPE DE COMUNICAÇÃO
\CargoNosCreditos{%
    Edson Ferreira Dias Junior	& \textbf{gerente de\newline \hfill Relações Institucionais}\\
    % \hfill dentro do cargo empurra "Relações Institucionais" para a direita
    % Cria um alinhamento especial dentro da célula da tabela
    Tais Souza	& \textbf{coordenadora}\\
    }
    
% DEMAIS MEMBROS DA EQUIPE DE COMUNICAÇÃO
\CargoNosCreditos{%
    Angelo Cuissi e Gislaine Bueno	& jornalistas\\
    % Dois jornalistas listados juntos
    
    Micael Melchiades e Paulo Mota	& \textit{designers}\\
    % \textit{} deixa "designers" em itálico
    % Dois designers listados juntos
    
   Jefferson Geraldo Rodrigues e\newline         % Quebra manual no meio dos nomes
    Viviane Doneda Martins Marigo	& profissionais de suporte\newline administrativo\\
    % Padrão similar à equipe anterior
    
   Anisa Feliciano e Mário Lemos	& estagiária/o de\newline Comunicação\\
   % Cargo quebrado em duas linhas
   
    Layza Vitoria Macedo Araújo	& jovem aprendiz\\
    % Função específica para jovem aprendiz
    }
\end{CreditoInstitucional}
\end{minipage}

% ESPAÇAMENTO VERTICAL ELÁSTICO
\vspace{\fill}  % Empurra o próximo conteúdo para o final da página
                % Sem asterisco (*), diferente do \vspace*{\fill} usado na capa

% SEÇÃO DE CRÉDITOS FINAIS E LICENÇA (SEM LAYOUT DE COLUNAS)
\begin{CreditoInstitucional}
\begin{center}  % Todo o conteúdo desta seção será centralizado

% CONFIGURAÇÃO DE ESPAÇAMENTO ENTRE LINHAS
\setasuspacing{\SingleSpacing}  % Define espaçamento simples entre linhas
                               % Comando personalizado para controlar espaçamento

% TÍTULO DA PUBLICAÇÃO
\color{crp1}      % Cor principal do CRP
\footnotesize     % Tamanho de fonte pequeno (menor que \small)
\textbf{Cartilha para produção de documentos escritos anticapacitistas}
\vspace{0.2\baselineskip}  % Pequeno espaço após o título

% SUBTÍTULO COM INFORMAÇÃO DA SUBCOMISSÃO
\color{black}     % Muda para cor preta
\scriptsize       % Tamanho ainda menor (menor que \footnotesize)
\textbf{Produzida pela Subcomissão Psicologia, Pessoa com Deficiência e Multiculturalidade do Conselho Regional de Psicologia de São Paulo}

% CRÉDITOS DE PRODUÇÃO
\fontebook        % Comando personalizado para fonte específica
\color{crp1}Projeto gráfico \color{black}Micael Melchiades
% Alterna cores: "Projeto gráfico" em cor CRP, nome em preto

\vspace{-0.618\baselineskip}  % Reduz espaço vertical (número áureo negativo)
\color{crp1}Preparação de texto \color{black}Angelo Cuissi
% Mesmo padrão: função em cor CRP, nome em preto

% ESPAÇAMENTO ANTES DA LICENÇA
\vspace*{1.618\baselineskip}  % Espaço baseado no número áureo (proporção dourada)
                              % Asterisco garante que o espaço seja mantido

% LOGO DA LICENÇA CREATIVE COMMONS
\includesvg[width=0.25\textwidth]{by-nc.svg}
% \includesvg = comando para incluir arquivos SVG (gráficos vetoriais)
% [width=0.25\textwidth] = largura de 25% da linha de texto
% {by-nc.svg} = arquivo da licença Creative Commons BY-NC

% TEXTO EXPLICATIVO DA LICENÇA
\scriptsize  % Texto pequeno para a explicação legal
Uma licença CC BY-NC foi atribuída a esta obra. São permitidas a remixagem,
a adaptação e a criação de novas obras com o conteúdo aqui publicado,
desde que seja dado o devido crédito à obra original e que a obra resultante
não seja usada para fins comerciais.
% Texto explicativo da licença Creative Commons Atribuição-NãoComercial

\end{center}             % Fecha o ambiente centralizado
\end{CreditoInstitucional}  % Fecha o ambiente de formatação de créditos

% ESPAÇAMENTO FINAL
\vspace*{\fill}  % Empurra qualquer conteúdo seguinte para o final da página
                 % Asterisco garante que funcione mesmo no final da página
